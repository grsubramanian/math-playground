\documentclass{article}
\usepackage[skip=10pt]{parskip}
\begin{document}

9. Any integer $n$ can be written as a product of prime powers. Say $n = \prod_{i = 1}^l {p_i}^{a_i}$. Then we can say that $f(n) = \prod_{i = 1}^l f({p_i}^{a_i})$, for any multiplicative function $f$, because the prime powers are relatively co-prime. Therefore, such a function is purely determined by its values on prime powers. This concludes the proof.

15a. Imagine a function $f(n) = 1$ for all values of $n$. Then, we can say that $\nu(n) = \sum_{d | n} 1 = \sum_{d | n} f(d)$. Thus, by the Mobius inversion theorem, $\sum_{d | n} \mu(n/d)\nu(d) = f(n) = 1$. This concludes the proof. 

15b. Imagine a function $f(n) = n$ for all values of $n$. Then, we can say that $\sigma(n) = \sum_{d | n} d = \sum_{d | n} f(d)$. Thus, by the Mobius inversion theorem, $\sum_{d | n} \mu(n/d)\sigma(d) = f(n) = n$. This concludes the proof.

\end{document}
