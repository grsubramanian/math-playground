\documentclass{article}
\usepackage[skip=10pt]{parskip}
\usepackage{amsmath}
\usepackage{amssymb}
\begin{document}

9. Any integer $n$ can be written as a product of prime powers. Say $n = \prod_{i = 1}^l {p_i}^{a_i}$. Then we can say that $f(n) = \prod_{i = 1}^l f({p_i}^{a_i})$, for any multiplicative function $f$, because the prime powers are relatively co-prime. Therefore, such a function is purely determined by its values on prime powers. This concludes the proof.

10. Consider $g(ab)$ where $gcd(a, b) = 1$. By definition of $g(n)$, $g(ab) = \sum_{d | ab} f(d)$. But since $gcd(a, b) = 1$, any divisor of $ab$ can be written in the form $d_ad_b$ where $d_a | a$ and $d_b | b$. Therefore, $g(ab) = \sum_{d_a | a, d_b | b} f(d_ad_b)$. But since $f$ is multiplicative, we get $g(ab) = \sum_{d_a | a, d_b | b} f(d_a)f(d_b) = (\sum_{d_a | a} f(d_a))(\sum_{d_b | b} f(d_b)) = g(a)g(b)$, which implies that $g$ too is multiplicative. This concludes the proof.

11. Consider $\mu(ab)$ where $gcd(a, b) = 1$. If either $a$ or $b$ is not square free, then both $\mu(ab) = 0 = \mu(a)\mu(b)$. If $a$ and $b$ are square free, then $ab$ is square free as well because $gcd(a, b) = 1$. And $\mu(ab)$ is $1$ or $-1$ depending on whether the parities of $\mu(a)$ and $\mu(b)$ match or not. Therefore $\mu(ab) = \mu(a)\mu(b)$, proving that $\mu(n)$ is a multiplicative function.

It is easy to see that if $f(n)$ is multiplicative, then both $nf(n)$ and $f(n)/n$ are multiplicative. We'll combine these lemmas with the result of problem 10 to conclude that the function $f(n) = n \sum_{d | n} \mu(d) / d$ is also a multiplicative function.

Now, by the results of problem 9, the values of any multiplicative function are purely determined by its values on prime powers. So, we'll evaluate $f(p^k)$ for some prime $p$ and some non-negative integer $k$. If we find that this value matches the value of $\phi(p^k)$ for all $k$, then we can infer that $\phi(n) = f(n) = n \sum_{d | n} \mu(d) / d$.

\begin{align*}
    f(p^k) &= p^k \sum_{d | p^k} \mu(d) / d \\
    f(p^k) &= p^k \sum_{i \in \{0 \cdots k\}} \mu(p^i) / p^i \\
    f(p^k) &= p^k ((1 / p^0) + (-1 / p^1) + 0 + 0 + ...) \\
    f(p^k) &= p^k (1 - 1 / p)
\end{align*}

Now, let's calculate $\phi(p^k)$ from first principles. Since the only integers in the range $\{1 \cdots p^k\}$ that are not coprime with $p^k$ are the multiples of $p$, and there are $p^{k - 1}$ such multiples, $\phi(p^k) = p^k = p^{k - 1} = p^k(1 - 1 / p)$. This concludes the proof.

12. From exercises 10 and 11, we have learned that $\mu(n)$ and $\phi(n)$ are multiplicative functions. Therefore, so are $\sum_{d | n} \mu(d) \phi(d)$, $\sum_{d | n} \mu(d)^2 \phi(d)^2$ and $\sum_{d | n} \mu(d) / \phi(d)$. This means that for any of these 3 functions, we can compute the value of the function for a given input $n$ by first computing the value of the function on all prime powers involved in $n$'s prime factorization, and then by multipliying these values. In all the equations below, we'll assume $n = p_1^{a_1} p_2^{a_2} \cdots p_k^{a_k}$.

\begin{align*}
    \sum_{d | p^k} \mu(d) \phi(d) &=  1(1) + (-1)(p - 1) + 0 + 0 + \cdots \\
    \sum_{d | p^k} \mu(d) \phi(d) &=  2 - p \\
    \therefore \sum_{d | n} \mu(d) \phi(d) &= \prod_{i = 1}^k (2 - p_i)
\end{align*}

\begin{align*}
    \sum_{d | p^k} \mu(d)^2 \phi(d)^2 &=  1^2(1^2) + (-1)^2(p - 1)^2 + 0 + 0 + \cdots \\
    \sum_{d | p^k} \mu(d)^2 \phi(d)^2 &=  p^2 - 2p + 2 \\
    \therefore \sum_{d | n} \mu(d)^2 \phi(d)^2 &= \prod_{i = 1}^k (p_i^2 -2p_i + 2)
\end{align*}

\begin{align*}
    \sum_{d | p^k} \mu(d) / \phi(d) &=  1 / 1 + (-1) / (p - 1) + 0 + 0 + \cdots \\
    \sum_{d | p^k} \mu(d) / \phi(d) &=  (p - 2) / (p - 1) \\
    \therefore \sum_{d | n} \mu(d) / \phi(d) &= \prod_{i = 1}^k (p_i - 2) / (p_i - 1)
\end{align*}

15a. Imagine a function $f(n) = 1$ for all values of $n$. Then, we can say that $\nu(n) = \sum_{d | n} 1 = \sum_{d | n} f(d)$. Thus, by the Mobius inversion theorem, $\sum_{d | n} \mu(n/d)\nu(d) = f(n) = 1$. This concludes the proof. 

15b. Imagine a function $f(n) = n$ for all values of $n$. Then, we can say that $\sigma(n) = \sum_{d | n} d = \sum_{d | n} f(d)$. Thus, by the Mobius inversion theorem, $\sum_{d | n} \mu(n/d)\sigma(d) = f(n) = n$. This concludes the proof.

16. Let's say that $n = p_1^{a_1} p_2^{a_2} \cdots p_l^{a_l}$. Then we know that $\nu(n) = \prod_{i = 1}^l (a_i + 1)$. We can infer that $\nu(n)$ is odd iff each of the terms on the RHS is odd i.e. if $a_i$ is even i.e. $n$ is a square. This concludes the proof.

17. Let's say that $n = p_1^{a_1} p_2^{a_2} \cdots p_l^{a_l}$. Then we know that $\sigma(n) = \prod_{i = 1}^l \sum_{j = 0}^{a_i} p_i^j$. We can infer that $\sigma(n)$ is odd iff each of the terms on the RHS is odd. i.e. $\sum_{j = 0}^{a_i} p_i^j$ is odd. If $p_i$ is odd, then this sum is odd iff $a_i$ is even. If $p_i$ is even (i.e. equal to $2$, the only even prime), then this sum is always odd irrespective of whether $a_i$ is even or odd. Thus, the condition is met iff $n$ is either a square or $2$ times a square. 
\end{document}
