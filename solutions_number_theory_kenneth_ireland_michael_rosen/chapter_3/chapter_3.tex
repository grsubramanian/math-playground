\documentclass{article}
\usepackage[skip=10pt]{parskip}
\usepackage{amsmath}
\usepackage{amssymb}
\begin{document}

6. Since $gcd(a_i, n) = 1$ for any $i$, and $gcd(a, n) = 1$, it is obvious that $gcd(aa_i, n) = 1$ for any $i$.

We next prove that $aa_i$ is not $\equiv aa_j \mod(n)$ if $i \ne j$. If not, $aa_i \equiv aa_j \mod(n)$, and $n | a(a_i - a_j)$. However, since $gcd(a, n) = 1$, this implies that $n | (a_i - a_j)$ or in other words $a_i \equiv a_j \mod(n)$, which is a contradiction.

This implies that if $a_1, a_2, \cdots a_{\phi(n)}$ is a reduced residue modulo $n$, so is $aa_1, aa_2, \cdots aa_{\phi(n)}$. This concludes the proof.

7. Consider the set of integers $< n$ that are coprime to $n$. Call these $a_1, a_2, \cdots a_{\phi(n)}$. Clearly these form a reduced residue system modulo $n$. Now for any $a$ such that $gcd(a, n) = 1$, we know from exercise 6 above that $aa_1, aa_2, \cdots aa_{\phi(n)}$ is also a reduced residue system modulo $n$.

Now, for any $i$, $aa_i = qn + r_i$, where $0 < r_i < n$ and $gcd(r_i, n) = 1$. In other words, $r_i = a_j$ for some $j$. Thus, for each $i$, there is a unique $j$ such that $aa_i \equiv a_j \mod(n)$.

Therefore,

\begin{align*}
    aa_1 aa_2 \cdots aa_{\phi(n)} & \equiv a_1 a_2 \cdots a_{\phi(n)} \mod(n) \\
    a^{\phi(n)} a_1 a_2 \cdots a_{\phi(n)} & \equiv a_1 a_2 \cdots a_{\phi(n)} \mod(n) \\
    \therefore a^{\phi(n)} & \equiv 1 \mod(n)
\end{align*}

where the last line is due to the fact that we can cancel out the term $a_1 a_2 \cdots a_{\phi(n)}$ on both sides due to it being relatively coprime to $n$. This concludes the proof. 

8. We know that $kx \equiv 1 \mod(p)$ must have exactly one unique solution since $gcd(k, p) = 1$. If the solution has the form $qp + r$, where $0 < r < p$, then $k(qp + r) \equiv 1 \mod(p)$, which leads to $kr \equiv 1 \mod(p)$. This proves that the there is some unique solution from the set $\{1, 2, \cdots (p - 1)\}$.

Further if $x = k$ is a solution for $kx \equiv 1 \mod(p)$, then $p | (k^2 - 1)$. i.e. $p | (k + 1)(k - 1)$. This is only possible if one of these terms is either $0$ or $p$, which happens iff $k = 1$ or $k = p - 1$.

This concludes the proof.

9. In light of the solution to problem 8, $(p - 1)!$ can be thought of as products of pairs of integers $k$ and $b_k$ such that $kb_k \equiv 1 \mod(p)$, except the terms $1$ and $(p - 1)$, which don't have the appropriate pairing. Therefore, $(p - 1)! \equiv 1 (p - 1) \mod(p)$ i.e. $(p - 1)! \equiv -1 \mod(p)$. This concludes the proof.

12. Consider $p \choose k$, where $p$ is prime, and $k \in \{1, 2, \cdots (p - 1)\}$. ${p \choose k} = p! / (k! (p - k)!)$ has the term $p$ in the numerator, which none of the terms in the denominator divide (given that they are all smaller than $p$, and $p$ is prime). Since ${p \choose k}$ is itself an integer, it must be an integer of the form $pq$ for some integer $q >= 1$, and is thus divisible by $p$.

By the binomial theorem,

\begin{align*}
               (a + 1)^p &= \sum_{i = 0}^p {p \choose i} a^i \\
                         &= (a^p + 1) + \sum_{i = 1}^{p - 1} {p \choose i} a^i \\
    \therefore (a + 1)^p &\equiv (a^p + 1) \mod(p)
\end{align*}

This concludes the proof.

15. The numerator of $1 + (1 / 2) + \cdots (1 / (p - 1))$ is obtained by multiplying the expression by $(p - 1)!$. The numerator can then be expressed as the sum $\sum_{k = 1}^{p - 1} (p - 1)! / k$.

First consider the very first term in this sum. It is equal to $(p - 1)!$ which is $\equiv -1 \mod(p)$ (due to problem 9, Wilson's theorem). Next consider the last term $(p - 1)! / (p - 1)$, which is $\equiv 1 \mod(p)$ (due to problem 9, Wilson' theorem, and because $(p - 1) \equiv -1 \mod(p)$). Thus, the first and last terms cancel each other out and become $\equiv 0 \mod(p)$.

Now, onto the other terms. Consider the $k$th term. Given problem 8's solution, we can say that the $k$th term is $\equiv (p - 1) b_k \mod(p)$, because we can pair of all other integers and reduce those multiples to be $\equiv 1 \mod(p)$. Putting it all together, we can say that the remaining terms are $\equiv (p - 1) \sum_{k = 2}^{p - 2} k \mod(p)$. Let's see what this simplifies to.

\begin{align*}
    \equiv & (p - 1) \sum_{k = 2}^{p - 2} k & \mod(p) \\ 
    \equiv & (p - 1) (((p - 1)(p - 2) / 2) - 1) & \mod(p) \\ 
    \equiv & -1 ((- 1)(- 2) / 2) - 1) & \mod(p) \\ 
    \equiv & 0 & \mod(p)
\end{align*}

Therefore, the numerator is divisible by $p$. This concludes the proof.
\end{document}
