\documentclass{article}
\usepackage[skip=10pt]{parskip}
\begin{document}

12. We'll try to find regular polygons that when stacked radially around a central vertex cover the whole $360^\circ$. For a given such regular polygon, assume that we stack $n$ copies. Then, the internal angle of that polygon must be equal to $(360/n)^\circ$. In other words, we are on a quest to finding out integers that divide $360$ and can be candidates for being internal angles of regular polygons. We can achieve this by doing two things: (a) listing all the factors of $360$ and finding those that can be written in the form $180 - 360/m$, which is the formula for the internal angle of a regular $m$ sided polygon. It turns out (although not demonstrated here on account of sheer laziness) that the only integers that satisfy this requirement are $60$, $90$ and $180$, which are of course the internal angles of equilateral trianges, squares and regular hexagons respectively. This concludes the proof.

13. Consider the set of all linear combinations of the form $x_1n_1 + x_2n_2 + ... + x_sn_s$. Clearly, the integers $abs(n_1)$, $abs(n_2)$, ... $abs(n_s)$ themselves belong to this set. Now consider the smallest positive integer $c$ that belongs to this set. Clearly, $c <= abs(n_1)$, $c <= abs(n_2)$, ... and $c <= abs(n_s)$. Consider the remainder $r$ obtained by dividing $abs(n_1)$ by $c$. We can write $r$ as $abs(n_1) - qc$ for some quotient $q$. Since $abs(n_1)$ and $c$ can both be written in the linear combination form above, so too can $r$. However, since $c$ is the smallest positive such linear combination, $r$ cannot be positive. In other words, we must have $r = 0$, or equivalently, $c | abs(n_1)$. We can similarly derive that $c | abs(n_2)$, ... and $c | abs(n_s)$. This tells us that $c$ is a common divisor of $n_1$, $n_2$, ... and $n_s$. In particular $c | d$, where $d = GCD(n_1, n_2, ...n_s)$. Since $c$ can be expressed as a linear combination of the above form, so too can $d$. This concludes the proof. In fact, since $d$ must divide all integers of the linear combination form above, we must have $d | c$, which leads us to the conclusion that $c = d$.

17. The proof is a special case of proof for exercise 18. Replace $m$ and $n$ both with $2$.

18. Assume that $\sqrt[n]{m}$ is rational i.e. can be expressed as a ratio of integers $a / b$. We'll prove that this leads to a contradiction. Raise both sides of the equation to the $n$th power. This yields $m = a^n / b^n$. The prime factorization of $a^n$ and $b^n$ must contain primes raised to an integer that is a multiple of $n$. Even as the primes in the numerator and denominator cancel out to yield $m$ (on the LHS), the resulting prime factorization must still contain primes raised to an integer that is a multiple of $n$. This implies that $m$ is the $n$th power of some integer, which contradicts what we have been told about $m$. This concludes the proof.

25. In this proof, we will leverage the equation derived from exercise 24, namely $x^n - y^n = (x - y)(x^{n-1} + x^{n-2}y + ... + y^{n-1})$. Firstly, $a^n - 1$ can only be prime if $a = 2$, because otherwise $a^n - 1$ which can be written as $(a - 1)(a^{n - 1} + a^{n - 2} + ... + 1)$, is a product of two integers both greater than $1$. Secondly, $2^n - 1$ can only be prime if $n$ itself is prime, because otherwise $n = ab$ for some $a > 1$ and $b > 1$, and $2^n - 1$, which can be written as $2^{ab} - 1 = (2^a)^b - 1 = (2^a - 1)((2^a)^{b - 1} + (2^a)^{b - 2} + ... + 1)$, is a product of two integers both greater than $1$. In summary, if $a^n - 1$ is prime, then $a = 2$ and $n$ is prime. This concludes the two proofs.

\end{document}
