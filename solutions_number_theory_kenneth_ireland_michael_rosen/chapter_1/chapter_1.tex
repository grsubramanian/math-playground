\documentclass{article}
\usepackage[skip=10pt]{parskip}
\begin{document}

8. As some upfront preparation, let's denote the integers $a / d$ as $a_d$ and $b / d$ as $b_d$, where $d = GCD(a, b)$. Now, onto the proof. Any solution to $ax + by = c$ must satisfy $ax + by = ax_0 + by_0$. Rearranging, we get $y = y_0 - (a / b)(x - x_0)$. For $y$ to be an integer, $(a / b)(x - x_0)$ must also be an integer, say $n$. In other words, $x = x_0 + nb / a = x_0 + nb_d / a_d$. Since $x$ is an integer, we must have $a_d | n$. Let's denote the integer $n / a_d$ as $r$. Thus, we get $x = x_0 + rb_d$. By similar argument, we can derive that $y = y_0 + la_d$ for some integer $l$. Let's now attempt to correlate $r$ and $l$ by expanding the expression $ax + by$ using these newly derived equations. $ax + by = d(a_dx + b_dy) = d(a_d(x_0 + rb_d) + b_d(y_0 + la_d)) = ax_0 + by_0 + da_db_d(r + l)$. The $ax + by$ term on the LHS gets canceled with the $ax_0 + by_0$ term on the RHS, and we get $r + l = 0$ or $r = -l$. Denote this integer as $t$. Plugging this back into the derived equations for $x$ and $y$, we get, $x = x_0 + tb_d$ and $y = y_0 - ta_d$. This concludes the proof.

12. We'll try to find regular polygons that when stacked radially around a central vertex cover the whole $360^\circ$. For a given such regular polygon, assume that we stack $n$ copies. Then, the internal angle of that polygon must be equal to $(360 / n)^\circ$. In other words, we are on a quest to finding out integers that divide $360$ and can be candidates for being internal angles of regular polygons. We can achieve this by doing two things: (a) listing all the factors of $360$ and finding those that can be written in the form $180 - 360 / m$, which is the formula for the internal angle of a regular $m$ sided polygon. It turns out (although not demonstrated here on account of sheer laziness) that the only integers that satisfy this requirement are $60$, $90$ and $180$, which are of course the internal angles of equilateral trianges, squares and regular hexagons respectively. This concludes the proof.

13. Consider the set of all linear combinations of the form $x_1n_1 + x_2n_2 + ... + x_sn_s$. Clearly, the integers $abs(n_1)$, $abs(n_2)$, ... $abs(n_s)$ themselves belong to this set. Now consider the smallest positive integer $c$ that belongs to this set. Clearly, $c <= abs(n_1)$, $c <= abs(n_2)$, ... and $c <= abs(n_s)$. Consider the remainder $r$ obtained by dividing $abs(n_1)$ by $c$. We can write $r$ as $abs(n_1) - qc$ for some quotient $q$. Since $abs(n_1)$ and $c$ can both be written in the linear combination form above, so too can $r$. However, since $c$ is the smallest positive such linear combination, $r$ cannot be positive. In other words, we must have $r = 0$, or equivalently, $c | abs(n_1)$. We can similarly derive that $c | abs(n_2)$, ... and $c | abs(n_s)$. This tells us that $c$ is a common divisor of $n_1$, $n_2$, ... and $n_s$. In particular $c | d$, where $d = GCD(n_1, n_2, ...n_s)$. Since $c$ can be expressed as a linear combination of the above form, so too can $d$. This concludes the proof. In fact, since $d$ must divide all integers of the linear combination form above, we must have $d | c$, which leads us to the conclusion that $c = d$.

17. The proof is a special case of proof for exercise 18. Replace $m$ and $n$ both with $2$.

18. Assume that $\sqrt[n]{m}$ is rational i.e. can be expressed as a ratio of integers $a / b$. We'll prove that this leads to a contradiction. Raise both sides of the equation to the $n$th power. This yields $m = a^n / b^n$. The prime factorization of $a^n$ and $b^n$ must contain primes raised to an integer that is a multiple of $n$. Even as the primes in the numerator and denominator cancel out to yield $m$ (on the LHS), the resulting prime factorization must still contain primes raised to an integer that is a multiple of $n$. This implies that $m$ is the $n$th power of some integer, which contradicts what we have been told about $m$. This concludes the proof.

25. In this proof, we will leverage the equation derived from exercise 24, namely $x^n - y^n = (x - y)(x^{n - 1} + x^{n - 2}y + ... + y^{n - 1})$. Firstly, $a^n - 1$ can only be prime if $a = 2$, because otherwise $a^n - 1$ which can be written as $(a - 1)(a^{n - 1} + a^{n - 2} + ... + 1)$, is a product of two integers both greater than $1$. Secondly, $2^n - 1$ can only be prime if $n$ itself is prime, because otherwise $n = ab$ for some $a > 1$ and $b > 1$, and $2^n - 1$, which can be written as $2^{ab} - 1 = (2^a)^b - 1 = (2^a - 1)((2^a)^{b - 1} + (2^a)^{b - 2} + ... + 1)$, is a product of two integers both greater than $1$. In summary, if $a^n - 1$ is prime, then $a = 2$ and $n$ is prime. This concludes the two proofs.

26. Firstly, $a^n + 1$ can only be prime if either $a = 1$ or $a$ were even, because with an odd $a$ such than $a > 1$, $a^n + 1$ would itself be even and greater than 2, so cannot be prime. The next part of this proof is along similar lines to that of exercise 25. This time, we will leverage another equation derived in exercise 24, namely that for odd $n$, $x^n + y^n = (x + y)(x^{n - 1} - x^{n - 2}y + x^{n - 3}y^2 ... + y^{n - 1})$. We see that $a^n + 1$ can only be prime if $n$ is a power of 2 because otherwise, $a^n + 1$, which can be written as $a^{st} + 1$ wherein $s$ is a power of $2$ and $t$ is an odd integer can be rewritten as $(a^s)^t + 1 = (a + 1)((a^s)^{t - 1} - (a^s)^{t - 2} + (a^s)^{t - 3} ... + 1)$, a product of two integers both greater than $1$. In summary, if $a^n + 1$ is prime, then $a$ is either $1$ or even, and $n$ is a power of $2$. This concludes the two proofs.

30. Assume that for some $n$ where $n > 1$, $H(n) = 1 / 2 + 1 / 3 + ... 1 / n$ is an integer equal to $M$. Now, consider the largest power of $2$, smaller than or equal to $n$. Call this $2^r$. Since we consider only cases where $n > 1$, we must have $r >= 1$, and thus $2^{r - 1}$ is a valid integer. Now, multiply both sides of the equation by $2^{r - 1}$. This yields $2^{r - 1}M = 2^{r - 1} / 2 + 2^{r - 1} / 3 + ... 2^{r - 1} / n$. Now let's investigate the outcome of simplifying all the ratios on the RHS. The term where the denominator is $2^r$ will become $1 / 2$. Further, because $2^r$ is the largest power of $2$ smaller than or equal to $n$, the denominators of all other terms will be stripped off all factors of $2$ from their prime factorization, and thus will be left being odd integers. Therefore, we get an equation of the form $2^{r - 1}M = a_1 / b_1 + a_2 / b_2 + ... + 1 / 2 + ... a_n / b_n$, wherein all terms on the LHS and RHS are in their simplified fractional forms, and have an odd denominator except for the term $1 / 2$ on the RHS. Rearraginging and simplifying, we get an equation of the form $a / b = 1 / 2$, where $a / b$ is a simplified fraction with $b$ odd. This is obviously a contradiction. And so, we see that $H(n) = 1 / 2 + 1 / 3 + ... 1 / n$ cannot be an integer for any $n$. This concludes the proof.

\end{document}
