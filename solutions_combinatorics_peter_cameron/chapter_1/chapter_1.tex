\documentclass{article}
\usepackage[skip=10pt]{parskip}
\usepackage{amsmath}
\begin{document}

2. Here's a solution to the Kirkman's problem for 9 schoolgirls walking for 4 days, represented in matrix form, wherein each matrix corresponds to a single day, and each row within a matrix corresponds to a single group.

The solution for day 1 lays the numbering scheme for the school girls. There is no magic in that. The solution for day 2 is just a transpose of the solution for day 1. Thereafter, in the remaining two days, the first column stays the same, and other columns just show up in various permutations of themselves.

Day 1:
\[
\begin{matrix}
    1 & 2 & 3\\
    4 & 5 & 6\\
    7 & 8 & 9
\end{matrix}
\]

Day 2:
\[
\begin{matrix}
    1 & 4 & 7\\
    2 & 5 & 8\\
    3 & 6 & 9
\end{matrix}
\]

Day 3:
\[
\begin{matrix}
    1 & 5 & 9\\
    2 & 6 & 7\\
    3 & 4 & 8
\end{matrix}
\]

Day 4:
\[
\begin{matrix}
    1 & 6 & 8\\
    2 & 4 & 9\\
    3 & 5 & 7
\end{matrix}
\]

\end{document}
