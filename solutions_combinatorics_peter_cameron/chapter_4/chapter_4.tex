\documentclass{article}
\usepackage[skip=10pt]{parskip}
\usepackage{amsmath}
\usepackage{amssymb}
\begin{document}

1a. Let's call the number of solutions for $n$ seating positions arranged in a straight line as $S(n)$. In any such solution, either the $n$th seat is chosen or not. The number of solutions for the latter case is $S(n - 1)$, and that for the former case is $S(n - 2)$ (because when the $n$th seat is chosen, the $(n - 1)$th seat cannot be). Thus, $S(n) = S(n - 1) + S(n - 2)$. For the base case, we can easily see that $S(0) = 1$ (because the act of not choosing anything is itself a solution) and $S(1) = 2$ (because we can either choose the only seat or not). Thus, we see that $S(n)$ has the same recurrence as $F(n)$, the Fibonacci sequence, but $S(0) = F(1)$ and $S(1) = F(2)$. Therefore, by induction, $S(n) = F(n + 1)$. This concludes the proof.

1b. Let's call the number of solutions for $n$ seating positions arranged in a circle as $T(n)$. We can see that almost any solution the problem where seats are arranged in a straight line can be a solution to the case where the seats are arranged in a circle. The only cases we must exclude are those where seat $1$ and seat $n$ are both chosen. For $n >= 4$, the number of such cases is $S(n - 4)$, since both $1$ and $n$ are chosen, neither seat $2$ nor seat $n - 1$ can be. So, the number of solutions for $n$ seats arranged in a circle is simply $S(n) - S(n - 4) = F(n + 1) - F(n - 3) = F(n) + F(n - 1) - F(n - 3) = F(n) + F(n - 2)$. For the cases $n = 3$ and $n = 2$, we can manually verify that this equation holds. This concludes the proof.

2a. Consider the two expressions $F_n^2 - F_{n + 1}F_{n - 1}$ and one of similar form obtained by replacing $n$ with $n - 1$, $F_{n - 1}^2 - F_nF_{n - 2}$. We will first prove that these two terms add to $0$.

\begin{align*}
    F_n^2 &- F_{n + 1}F_{n - 1} + F_{n - 1}^2 &- F_nF_{n - 2} \\
    F_n^2 &- F_{n - 1}(F_{n + 1} - F_{n - 1}) &- F_nF_{n - 2} \\
    F_n^2 &- F_{n - 1}F_n &- F_nF_{n - 2} \\
    &F_n(F_n - F_{n - 1}) &- F_nF_{n - 2} \\
    &F_nF_{n - 2} &- F_nF_{n - 2} \\
    0
\end{align*}

Thus, such expressions have the same magnitude but alternate signs each step along the way (from $n = 0$ onwards). We can check the base case value of $F_1^2 - F_2F_0 = -1$ to infer that $F_n^2 - F_{n + 1}F_{n - 1} = (-1)^n$. This concludes the proof.

2b. This one can be derived by repeated application of the recursive definition of the Fibonacci series.

\begin{align*}
    F_{n + 2} &= F_n + F_{n + 1} & \\
    F_{n + 2} &= F_n + F_{n - 1} + F_n \\
    F_{n + 2} &= F_n + F_{n - 1} + F_{n - 2} + F_{n - 1} \\
    \cdots \\
    F_{n + 2} &= \sum_{i = n}^0F_i + F_1
\end{align*}

This concludes the proof.

2c. We can easily see that $F_2 = F_0^2 + F_1^2 = 2$ and $F_3 = F_1(F_0 + F_2) = 3$. Assume that the given equations are true for up to $2n + 1$. Then, we can derive the following for $F_{2n + 2}$ and $F_{2n + 3}$.

\begin{align*}
    F_{2n + 2} &= F_{2n} + F_{2n + 1} \\
    F_{2n + 2} &= F_{n - 1}^2 + F_n^2 + F_n(F_{n - 1} + F_{n + 1}) \\
    F_{2n + 2} &= F_{n - 1}^2 + F_n(F_n + F_{n - 1} + F_{n + 1}) \\
    F_{2n + 2} &= F_{n - 1}^2 + F_n(2F_{n + 1}) \\
    F_{2n + 2} &= (F_{n + 1} - F_n)^2 - 2F_nF_{n + 1} \\
    F_{2n + 2} &= F_n^2 + F_{n + 1}^2
\end{align*}

\begin{align*}
    F_{2n + 3} &= F_{2n + 1} + F_{2n + 2} \\
    F_{2n + 3} &= F_n(F_{n - 1} + F_{n + 1}) + F_n^2 + F_{n + 1}^2 \\
    F_{2n + 3} &= F_n(F_{n - 1} + F_{n + 1} + F_n) + F_{n + 1}^2 \\
    F_{2n + 3} &= F_n(2F_{n + 1}) + F_{n + 1}^2 \\
    F_{2n + 3} &= F_{n + 1}(2F_n + F_{n + 1}) \\
    F_{2n + 3} &= F_{n + 1}(F_n + F_{n + 2}) \\
\end{align*}

This concludes the proof by induction.

7. The inefficient algorithm computes $F_n$ recursively as $F_{n - 1} + F_{n - 2}$, then each of these recursively without storing intermediate results. Let's call the number of additions required to compute $F_n$ as $A_n$. Clearly, $A_0 = A_1 = 0$, since the algorithm uses the base case values of $F_0 = F_1 = 1$ and does not need to perform any additions in this case. Further, we can see that $A_n = A_{n - 1} + A_{n - 2} + 1$, because in order to compute $F_n$ naively, the algorithm would have to compute $F_{n - 1}$ and $F_{n - 2}$ independently, which would respectively take $A_{n - 1}$ and $A_{n - 2}$ additions, and then finally perform an extra addition to sum these two values. It is easy to see via induction that $A_n = F_n - 1$. This concludes the proof.

9b. Let's refer to the number of ways in which the positive integer $n$ can be written as a summation of positive integers (wherein ordering of summands is relevant) as $S(n)$. We can easily see that $S(1) = 1$. For any other $n$, the last summand in any such summation is one of $\{1, 2, \cdots n - 1\}$. In case the last summand is $k$, the rest of the sum $n - k$ can be written (by definition) in $S(n - k)$ ways. Therefore, we can see that $S(n) = \sum_{k = 1}^{n - 1} S(n - k)$. Since the first term is $S(n - 1)$ and the sum of the remaining terms is also $S(n - 1)$ (by recursion), we get $S(n) = S(n - 1) + S(n - 1) = 2S(n - 1)$. It is easy to see via induction that $S(n) = 2^{n - 1}$. This concludes the proof.

10. If $n$ is odd, wee see that

\begin{align*}
    f(n + 2) - f(n + 1) &= 2f(n + 1) + 1 - (2f(n)) \\
    f(n + 2) - f(n + 1) &= 2(f(n + 1) - f(n)) + 1 \\
    f(n + 2) - f(n + 1) &= 2(2f(n) - f(n)) + 1 \\
    f(n + 2) - f(n + 1) &= 2f(n) + 1
\end{align*}

If $n$ is even, we see that

\begin{align*}
    f(n + 2) - f(n + 1) &= 2f(n + 1) - (2f(n) + 1) \\
    f(n + 2) - f(n + 1) &= 2(f(n + 1) - f(n)) - 1 \\
    f(n + 2) - f(n + 1) &= 2(2f(n) + 1 - f(n)) - 1 \\
    f(n + 2) - f(n + 1) &= 2f(n) + 1
\end{align*}

This concludes the proof for the first part. For the second part, where we want to find an expression for $f(n)$ as a function of $n$, we start by noting that $f(1) = 1$ (because this is given) and $f(2) = 2$ (by a simple application of the provided recurrence formula). Let's now consider $n = 2k$, an even integer, and observe what the recurrence formula teaches us.

\begin{align*}
    f(2k + 1) &= 2f(2k) + 1 \\
    f(2k + 1) &= 2(2f(2k - 1)) + 1 \\
    f(2k + 1) &= 4f(2k - 1) + 1 \\
    \therefore f(2k + 1) &= (4^{k + 1} - 1) / 3 \\
    \therefore f(n) &= (2^{n + 1} - 1) / 3
\end{align*}

and

\begin{align*}
    f(2k) &= 2f(2k - 1) \\
    f(2k) &= 2(2f(2k - 2) + 1) \\
    f(2k) &= 4f(2k - 2) + 2 \\
    \therefore f(2k) &= 2(4^k - 1) / 3 \\
    \therefore f(n) &= (2^{n + 1} - 2) / 3
\end{align*}

The last line of the above two derivations is based on the sum of the geometric series $\sum_{i = 0}^r 4^r = (4^{r + 1} - 1) / 3$. For the "even" series, this works out just cleanly because $f(2) = 2$.

11a. Any valid sequence must either end with $n$ or not. If it does end with $n$, the previous term (and all other terms before it) in the sequence must be in the range $\{1 \cdots \lfloor {n / 2} \rfloor \}$. If it does not end with $n$, then all terms in the sequence must be in the range $\{1 \cdots n - 1\}$. Therefore, $s(n) = s(n - 1) + s(\lfloor {n / 2} \rfloor)$. $s(0) = 1$ because the only possible sequence is the empty sequence. Following table shows some initial values.

\begin{center}
    \begin{tabular}{c c}
        \hline
        n & s(n) \\
        \hline
        0 & 1 \\
        1 & 2 \\
        2 & 4 \\
        3 & 6 \\
        4 & 10 \\
        5 & 14 \\
        6 & 20 \\
        7 & 26 \\
        8 & 36 \\
        9 & 46 \\
        10 & 60
    \end{tabular}
\end{center}

Now, let's consider the generating function $G(t) = \sum_{k = 0}^\infty s(k)t^k$. Using the above recursion, we can express it as follows.


\begin{align*}
    G(t) &= s(0) + \sum_{k = 1}^\infty s(k - 1)t^k + \sum_{k = 1}^\infty s(\lfloor {k / 2} \rfloor)t^k \\
\end{align*}

For the first infinite sum on the RHS, we can replace $k - 1$ by another variable $l$ and get $\sum_{l = 0}^\infty s(l)t^{l + 1} = tG(t)$.

For the second infinite sum on the RHS, we need to do some more manipulation to simplify it.

\begin{align*}
    \sum_{k = 1}^\infty s(\lfloor {k / 2} \rfloor)t^k &= s(0)t + s(1)t^2 + s(1)t^3 + s(2)t^4 + s(2)t^5 ... \\
    \sum_{k = 1}^\infty s(\lfloor {k / 2} \rfloor)t^k &= (s(0)t + s(1)t^3 + s(2)t^5 + ...) + (s(1)t^2 + s(2)t^4 + ...) \\
    \sum_{k = 1}^\infty s(\lfloor {k / 2} \rfloor)t^k &= t(s(0) + s(1)t^2 + s(2)t^4 + ...) + (-s(0)) + (s(0) + s(1)t^2 + s(2)t^4 + ...) \\
    \sum_{k = 1}^\infty s(\lfloor {k / 2} \rfloor)t^k &= tG(t^2) + (-s(0)) + G(t^2) \\
    \sum_{k = 1}^\infty s(\lfloor {k / 2} \rfloor)t^k &= (1 + t)G(t^2) -s(0) \\
\end{align*}

Combining everything, we get

\begin{align*}
    G(t) &= s(0) + tG(t) + (1 + t)G(t^2) - s(0) \\
    \therefore (1 - t)G(t) &= (1 + t)G(t^2)
\end{align*}

This concludes the proof.

\end{document}
