\documentclass{article}
\usepackage[skip=10pt]{parskip}
\usepackage{amsmath}
\usepackage{amssymb}
\begin{document}

3a. Consider any processing operation that has two types A and B. Now, let's count the number of ways we can pick $k$ objects from $n$ and apply processing operation A on $l$ of the $k$ objects and operation B on the remaining $k - l$ objects. We can count this in two equivalent ways.

\begin{itemize}
    \item ${n \choose k}{k \choose l}$, which involves picking $k$ objects from $n$, and then picking the $l$ from amongst those $k$.
    \item ${n \choose l}{n - l \choose k - l}$, which involves picking $l$ objects from $n$, and then picking the $k - l$ from amongst the remaining $n - l$.
\end{itemize}

This concludes the proof.

3b. Let's say we have $m$ boys and $n$ girls to pick a total of $k$ people from. We can count this in two equivalent ways.

\begin{itemize}
    \item ${m + n \choose k}$, which involves picking $k$ objects from amongst the total of $m + n$ people.
    \item $\sum_{i=0}^k {m \choose i}{n \choose k - i}$, which involves counting $k + 1$ exclusive possibilities, wherein in each possibility, we pick a certain number $i$ of boys from amongst the $m$ boys, and the remaining $k - i$ required people from the $n$ girls.
\end{itemize}

This concludes the proof.

3c. This can easily be proven by repeated application of the rule ${n + k + 1 \choose k} = {n + k \choose k} + {n + k \choose k - 1}$, to the last term in this equation all the way down to ${n \choose 0}$. Induction seems like the cleanest way to setup this proof.

3d. The algebraic proof is based on differentiating the two sides of the binomial expression for $(1 + t)^n$ and then replacing $t$ with $1$. But this is a bit mundane. So let's give a combinatorial proof instead.

Consider $n$ objects, and say we need to do two things - first pick at least one of them, and then designate a leader from amongst those picked. We can count this in two equivalent ways.

\begin{itemize}
    \item $\sum_{i=1}^k {n \choose i} i$, which involves counting $k$ exclusive possibilities, wherein in each possibility, we first pick a certain number $i$ of objects from among the $n$ objects, and then designate one of those $i$ objects as leader.
    \item $n2^{n - 1}$, which involves first picking a leader, and then picking $0$ or more objects from the remaining $n - 1$ objects. 
\end{itemize}

This concludes the proof.

3e. We'll use the binomial theorem. Consider the expression $(1 + a)^n(1 - a)^n = (1 - a^2)^n$ and its full expansion. We'd like to specifically find the coefficient of the term $a^n$ in the expansion. We can count this in two equivalent ways.

\begin{itemize}
    \item by picking for each value of $k$ in the range $[0, n]$, $a^k$ from the expansion of $(1 + a)^n$ and multiplying it with $a^{n - k}$ from the expansion of $(1 - a)^n$, thus implying that the desired coefficient would be $\sum_{k=0}^n{n \choose k}{n \choose n - k} = \sum_{k=0}^n{n \choose k}{n \choose k}$.
    \item by picking the coefficient of $a^n$ in the expansion of $(1 - a^2)^n$, which happens to be $0$ in case $n$ is odd, and $(-1)^{n/2}{n \choose n / 2}$ in case $n$ is even.
\end{itemize}

This concludes the proof.

6. For any prime $p$, it is clear that $0^p \equiv 0 \mod{p}$. Let's assume that $r$ is the largest integer for which we know $r^p \equiv r \mod{p}$ to be true. Then, the following is true.

\begin{align*}
    (1 + r)^p &\equiv (1 + r^p) \mod{p} \\
    (1 + r)^p &\equiv (1 + r) \mod{p}
\end{align*}

This concludes the proof by induction.

8. Any cyclic permutation on $n$ objects can be mapped to arrangements of those objects - simply pick a starting object and then apply the permutation to obtain the next object in the arrangement. Since there are $n$ choices for the starting object, there are $n$ resulting arrangements. Conversely, given any arrangement, it is possible to reverse engineer a cyclic permutation which when expressed as above can yield the arrangement. Thus, the number of cyclic permutations on $n$ objects is equal to the number of equivalence classes in the set of arrangements of $n$ objects, wherein two arrangements are equivalent if one can be obtained from another by cyclically shifting the arragenement. Since the total number of arrangements is $n!$, and the the size of each equivalence class is $n$, the number of equivalence classes, and thus the desired count is $(n - 1)!$. This concludes the proof.

9. Consider any permutation. It must have at least one cycle in it. In general, such a cycle has the form $a_0, a_1, \cdots a_{k - 1}$, for some $k >= 1$. By definition, applying the permutation $\pi$ causes for each $i \in \{0 \cdots k - 1\}$ the object at position $a_i$ to move to position $\pi(a_i) = a_{(i + 1) \% k}$. In other words, each application of the permutation is equivalent to cyclically right shifting the array whose original value is $a_0, a_1, \cdots a_{k - 1}$. Since all items in this array are distinct, it must take exactly $k$ such consecutive cyclic right shifts to cause the array to become identical to the original array. Therefore, the order of this cycle is $k$.

Each cycle in a permutation acts independently of other cycles. Thus, the order of one cycle is completely independent of the order of another cycle in the same permutation. Now, we can see that repeatedly applying a permutation on an arrangement yields the original arrangement after $M$ applications where $M$ is a multiple of the orders of the individual cycles in the permutation. In particular, the order of the permutation must be the least such multiple. 

11. Let's denote $k = n - m$, the count of non-repeating letters in the provided letter string. The number of generatable strings that contain $l$ instances of the repeated letter and some $r$ of the non-repeating letters is ${k \choose r} (l + r)! / l!$. This is because we can choose $r$ of the non-repeating letters from the $k$ non-repeated letters in $k \choose r$ ways, then arrange the total $l + r$ letters in $(l + r)! / l!$ ways because $l$ of the letters are identical. The desired result is then simply the sum of these counts for all values of $l$ and $r$ in the range $\{0 \cdots m\}$ and $\{0 \cdots k\}$ respectively.

\end{document}
