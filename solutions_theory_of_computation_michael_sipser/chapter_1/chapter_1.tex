\documentclass{article}
\usepackage[skip=10pt]{parskip}
\usepackage{amsmath}
\usepackage{amssymb}
\begin{document}

31. We'll prove that the perfect shuffle of two regular languages $A$ and $B$ is regular by constructing a DFA that recognizes it. Let's say that $A$ and $B$ are recognized respectively by the DFAs $D_A = (Q_A, \Sigma, \delta_A, q_A, F_A)$ and $D_B = (Q_B, \Sigma, \delta_B, q_B, F_B)$. We construct a DFA $D_S$ as follows.

\begin{align*}
    Q_S                     &= Q_A \times Q_B \times \{A, B\} \\
    \Sigma_S                &= \Sigma \\
    \delta_S((q, q', A), c) &\rightarrow (\delta_A(q, c), q', B) \\
    \delta_S((q, q', B), c) &\rightarrow (q, \delta_B(q', c), A) \\
    q_S                     &= (q_A, q_B, A) \\
    F_S                     &= F_A \times F_B \times \{A\}
\end{align*}

$D_S$ tracks the progress of the input string through both $D_A$ and $D_B$ by keeping track of the states of these DFAs, and alternating between their transition function upon reading each symbol. $D_S$ starts out tracking the start states of both DFAs, and expecting to effect the next transition on $D_A$. $D_S$ reaches a final state upon reading the input string iff the input string has a form $a_1 b_1 a_2 b_2 \cdots a_k b_k$ where $a_1 a_2 \cdots a_k \in L(D_A)$ and $b_1 b_2 \cdots b_k \in L(D_B)$, because such and only such strings can take the state of $D_S$ to some state in $F_A \times F_B \times \{A\}$. 

This concludes the proof.

33. We'll prove that $DROPOUT(A)$ is a regular language by constructing an NFA that recognizes it. Since $A$ is a regular language, some DFA $D = (Q, \Sigma, \delta, q_0, F)$ recognizes it. We construct NFA $N$ as follows.

\begin{align*}
    Q_N                   &= Q \cup copy(Q) \\
    \Sigma_N              &= \Sigma \\
    \delta_N(q, c)        &\rightarrow \delta(q, c) \\
    \delta_N(q, \epsilon) &\rightarrow copy(q) \\
    \delta_N(copy(q), c)  &\rightarrow copy(\delta(q, c)) \\
    q_{0N}                &= q_0 \\
    F_N                   &= copy(F)
\end{align*}

For every state $q$ in $Q$, we'll create a copy state $copy(q)$. Call the set of all copy states $copy(Q)$. We'll augment the transition function as follows. Firstly, we'll add identical transitions within $copy(Q)$ to what existing in $Q$. Secondly, we'll add $\epsilon$ transitions from $Q$ to $copy(Q)$. The latter makes skipping a symbol equivalent to transitioning from some state $q$ to $copy(q)$. The idea is that once the NFA has skipped a symbol, it enters some state in $copy(Q)$ and stays within $copy(Q)$ till the very end, since there is no way to go back into $Q$ from $copy(Q)$. And further that unless the NFA skips a symbol (through an $\epsilon$ transition from $Q$ to $copy(Q)$), it won't enter a state in $copy(Q)$ and thus won't be able to accept the input because $F_N = copy(F) \subseteq copy(Q)$. Therefore, $N$ recognizes $DROPOUT(A)$.

This concludes the proof.

43. We need to prove two things, firstly that for each regular language, we can construct an all-NFA that recognizes it, and secondly that any all-NFA recognizes a regular language. The first part is clear because for each regular language, we can construct a DFA, and a DFA is also an all-NFA. We prove the second part below by constructing a DFA equivalent to any given all-NFA.

Given an all-NFA $(Q, \Sigma, \delta, q_0, F)$, construct an equivalent DFA $D$ as follows. We'll assume that $F$ is not empty, otherwise, it is trivial to construct an equivalent DFA, one that doesn't accept anything.

\begin{align*}
    Q_D                                                  &= P(Q) \\
    \Sigma_D                                             &= \Sigma \\
    \delta_D((\{q_{a_1}, q_{a_2}, \cdots q_{a_k}\}), c)  &\rightarrow (\cup_{i=1}^k eps(\delta(q_{a_i}, c))) \\
    q_{0D}                                               &= (eps(q_0)) \\
    F_D                                                  &= P(F) \setminus \{\}
\end{align*}

This construction is very similar to the one used to prove that NFAs are equivalent to DFAs. We create a separate state in the DFA to represent each subset of states of the all-NFA (to represent the idea that the all-NFA could be in any subset of its states at a given point of time in its execution). The transition function of the DFA builds on the transition function of the all-NFA in such a manner that when the all-NFA transits from some subset of states to another subset of states upon a certain symbol, the DFA transits from the state that represents the first subset to the state that represents the $\epsilon$ closure of the second subset. The DFA starts out the state that represents the $\epsilon$ closure of the start state. And finally, by making sure that the states of the DFA that correspond to the subsets of the powerset of the all-NFA's final states (except the empty set) are the final states of the DFA, we ensure that the DFA accepts a string if an only if the all-NFA accepts it.

This concludes the proof.

\end{document}
