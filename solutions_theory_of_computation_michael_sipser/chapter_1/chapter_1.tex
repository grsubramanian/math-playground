\documentclass{article}
\usepackage[skip=10pt]{parskip}
\usepackage{amsmath}
\usepackage{amssymb}
\begin{document}

31. We'll prove that the perfect shuffle of two regular languages $A$ and $B$ is regular by constructing a DFA that recognizes it. Let's say that $A$ and $B$ are recognized respectively by the DFAs $D_A = (Q_A, \Sigma, \delta_A, q_A, F_A)$ and $D_B = (Q_B, \Sigma, \delta_B, q_B, F_B)$. We construct a DFA $D_S$ with the following 5-tuple.

\begin{align*}
    Q_S                     &= Q_A \times Q_B \times \{A, B\} \\
    \Sigma_S                &= \Sigma \\
    \delta_S((q, q', A), c) &\rightarrow (\delta_A(q, c), q', B) \\
    \delta_S((q, q', B), c) &\rightarrow (q, \delta_B(q', c), A) \\
    q_S                     &= (q_A, q_B, A) \\
    F_S                     &= F_A \times F_B \times \{A\}
\end{align*}

$D_S$ tracks the progress of the input string through both $D_A$ and $D_B$ by keeping track of the states of these DFAs, and alternating between their transition function upon reading each symbol. $D_S$ starts out tracking the start states of both DFAs, and expecting to effect the next transition on $D_A$. $D_S$ reaches a final state upon reading the input string iff the input string has a form $a_1 b_1 a_2 b_2 \cdots a_k b_k$ where $a_1 a_2 \cdots a_k \in L(D_A)$ and $b_1 b_2 \cdots b_k \in L(D_B)$, because such and only such strings can take the state of $D_S$ to some state in $F_A \times F_B \times \{A\}$. 

This concludes the proof.

\end{document}
