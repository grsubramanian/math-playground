\documentclass{article}
\usepackage[skip=10pt]{parskip}
\usepackage{amsmath}
\usepackage{amssymb}
\begin{document}

12. Given that $L$ is Turing recognizable, some enumerator $E$ enumerates $L$. Consider another TM $D$ that operates as follows. For a given input $i$, $D$ runs $E$ until it prints $<M_i>$, the $i$th item. Then $D$ runs $M_i$ on input $i$. Finally, if $M_i$ accepts, $D$ rejects, and if $M_i$ rejects, $D$ accepts. Clearly, $D$ always terminates on all inputs, and is thus a decider itself. Let's call the (decidable) language of $D$ as $L_D$.

Now, let's assume that there is some $<M> \in L$ such that $M$ decides $L_D$. Then $<M>$ must appear at some index $k$ in the enumeration done by $E$. Now, since $M$ and $D$ are both deciders for $L_D$, $M$ must accept input $k$ iff $D$ accepts input $k$. However, by definition, $D$ rejects $k$ if $M$ accepts $k$. Therefore, our supposition that such a TM $M$ exists is wrong, which implies that the language $L_D$ has no decider in $L$. This concludes the proof.

20. Consider the following transformation procedure on any DFA $D$.

\begin{enumerate}
    \item Create an NFA $N^r$ as follows.
    \item Copy over all states from $D$ into $N^r$.
    \item If there exists a transition $\delta_D(q, a) \rightarrow q'$ in $D$, add correspondingly the transition $\delta_{N^r}(q', a) \rightarrow q$. This is visually equivalent to reversing the transition edges.
    \item Add a new state $s$ and mark it as a start state of $N^r$. Add transitions of the form $\delta_{N^r}(s, \epsilon) \rightarrow f$ for each state $f$ that is an accept state in $D$.
    \item Make the start state of $D$ the only accept state of $N^r$.
    \item Finally, convert the NFA $N^r$ into an equivalent DFA $D^r$.
\end{enumerate}

Now, we can see from the above construction that $N^r$ (and thus also $D^r$) accepts $w^r$ iff $D$ accepts $w$. Therefore, $N^r$ (and thus also $D^r$) accepts both $w^r$ and $w$ iff $D$ too accepts both $w$ and $w^r$. Taking this to its conclusion, we can see that $N^r$ (and thus also $D^r$) and $D$ are equivalent iff for any string $w$, $D$ accepts $w^r$ iff it accepts $w$.

Now, consider a TM $M$ that for any given DFA description $<d>$, performs the above DFA transformation procedure on $d$, and then checks whether $d$ and $d^r$ are equivalent. The transformation procedure itself takes a finite number of steps. Further, we know (from chapter 4 text) that the problem of determining whether two DFAs are equivalent is decidable, and thus can also be done in a finite number of steps. Therefore the TM $M$ is a decider. In fact, it decides the language $S = \{ <D> \mid D \text{ is a DFA that accepts } w^r \text{ iff it accepts } w \}$. This concludes the proof.

22. Given that $A$ and $B$ are co-Turing recognizable, there are recognizers for $\overline{A}$ and $\overline{B}$. Call these $M$ and $N$. Consider a TM $T$ that operates as follows. For a given input $w$, it runs $M$ and $N$ one step at a time taking turns, and accepts if $N$ accepts first, else rejects. Because $A$ and $B$ are disjoint, either $M$ or $N$ terminates on any given input. Therefore $T$ is a decider. Call the language of $T$ as $C$.

Now, we need to prove two things. Firstly, we need to prove that $A \subseteq C$. Consider the execution of $T$ on any input $w \in A$. In that execution, $N$ accepts the input and $M$ rejects that input (or loops forever). Finally, $T$ accepts the input, and thus $w \in C$. Therefore $A \subseteq C$.

Secondly, we need to prove that $B \subseteq \overline{C}$. Consider the execution of $T$ on any input $w \in \overline{B}$. In that execution, $N$ accepts the input. If it accepts first, $T$ accepts and $w \in C$ else $T$ rejects and $w \notin C$. Therefore, $C \subseteq \overline{B}$. Since $A$ and $B$ are disjoint, this implies that $B \subseteq \overline{C}$.

This implies that the decidable language $C$ (decided by $T$) divides $A$ and $B$, thus concluding the proof.

\end{document}
